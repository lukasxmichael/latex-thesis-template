\chapter*{Abstract}

Despite the documented existence of new issue premiums in credit markets, no systematic machine learning approach has been developed to identify these opportunities in European investment-grade corporate bond markets, where traditional selection methods rely heavily on subjective judgment and experience-based analysis. This research addresses this gap by developing the first comprehensive framework that transforms bond characteristics and macroeconomic conditions into actionable investment signals using extreme gradient boosting (XGBoost).

Through an analysis of European investment grade bond issuances from 2000 to 2025, the study demonstrates that there are systematic patterns in the pricing of new issues that can be exploited quantitatively, even within relatively efficient markets. The machine learning framework achieves high precision in identifying outperforming bonds while generating meaningful risk-adjusted returns with consistent performance under varying market conditions. Key predictive features include first-time issuer status, credit risk indicators, and macroeconomic factors such as market direction and inflation dynamics.

These findings establish that quantitative methods can meaningfully improve traditional investment decision-making in European corporate bond markets. The framework provides portfolio managers with a systematic tool for alpha generation while maintaining appropriate risk controls, potentially transforming how fixed income professionals approach the evaluation and selection of new issues in institutional investment environments.