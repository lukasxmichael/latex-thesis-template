\chapter*{AI Declaration}
\thispagestyle{empty}

This bachelor thesis was completed with the assistance of several artificial intelligence tools that supported different aspects of the research and writing process. The primary tools employed were ChatGPT in two versions (o3 for conceptual explanations and idea generation, o4-mini-high for coding assistance), Claude AI (Sonnet 3.7) and Quilbot for language enhancement, and Elicit and Perplexity (Pro) for academic source identification.

These tools served distinct functions throughout the thesis development. ChatGPT o3 helped clarify complex theoretical concepts and generate initial ideas during the early research phases. The o4-mini-high variant proved helpful for debugging code. Claude AI and Quilbot enhanced the clarity and academic tone of written content. Elicit and Perplexity Pro accelerated the literature review process by identifying relevant academic papers and sources more efficiently than traditional database searches alone.

The integration of these AI tools significantly enhanced productivity across multiple dimensions of the research process. Coding efficiency improved substantially through automated error identification and debugging assistance, reducing the time required for technical implementation. Source identification became considerably faster, particularly for discovering interdisciplinary connections and recent publications that might have been overlooked through conventional search methods.

However, the experience also revealed important limitations that required careful oversight. AI tools occasionally suggested irrelevant sources that needed independent verification before incorporation into the research. More significantly, the tools demonstrated limited utility for highly specialized tasks, particularly data extraction procedures requiring specific API implementations. These technical limitations became apparent when addressing novel methodological requirements that fell outside the scope of the AI systems' training data.