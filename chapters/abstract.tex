\cleardoubleoddpage

\chapter*{Abstract}
\thispagestyle{empty} %hide page numbers

Current research explains the general existence of the new issue premium in credit markets in terms of structural underpricing, followed by strong price appreciation. This research builds on existing literature by investigating whether machine learning algorithms can systematically predict short-term outperformance of European investment-grade corporate bonds using explanatory variables available at issuance. Through comprehensive feature engineering incorporating both microeconomic bond characteristics and macroeconomic market conditions, an XGBoost ensemble classifier is developed and optimized using investment-focused objective functions rather than traditional statistical accuracy measures. The model analyzes 7,320 European investment-grade bond issuances from 2000 to 2025, achieving 72\% precision in identifying outperforming bonds while maintaining a conservative 3.1\% market participation rate. When implemented as a systematic investment strategy over the evaluation period from January 2024 through April 2025, the approach generates an average active return of 36 basis points over five-day investment horizons with strong risk-adjusted characteristics including a Sharpe-like ratio of 0.81 and 75\% win rate. The results demonstrate that machine learning can effectively enhance traditional new issue selection methodologies, providing portfolio managers with systematic tools for alpha generation in European investment-grade corporate bond markets.